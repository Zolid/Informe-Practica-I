% Template:     Informe/Reporte LaTeX
% Documento:    Archivo de ejemplo
% Versión:      4.3.6 (30/07/2017)
% Codificación: UTF-8
%
% Autor: Pablo Pizarro R.
%        Facultad de Ciencias Físicas y Matemáticas
%        Universidad de Chile
%        pablo.pizarro@ing.uchile.cl, ppizarror.com
%
% Manual template: [http://latex.ppizarror.com/Template-Informe/]
% Licencia MIT:    [https://opensource.org/licenses/MIT/]

% NUEVA SECCIÓN
% Las secciones se inician con \section, si se quiere una sección sin "número" se pueden usar las funciones \sectionanum (sección sin número) o la función \sectionanumnoi para crear el mismo título sin numerar y sin aparecer en el índice
\section{Introducción}
	
	% SUB-SECCIÓN
	% Las sub-secciones se inician con \subsection, si se quiere una sub-sección sin "número" se pueden usar las funciones \subsectionanum (nuevo subtítulo sin numeración) o la función \subsectionanumnoi para crear el mismo subtítulo sin numerar y sin aparecer en el índice
	\subsection{Lugar de Trabajo}
		
		\newpar{La prática profesional se realizó en el Web Intelligence Centre \cite{ref1}, en adelante WIC, es un centro de investigación depediente de la Facultad de Ciencias Físicas y Matemáticas de la Universidad de Chile, cuya misión es desarrollar investigación en el campo de Tecnologías de Información creando soluciones para abordar problemas complejos de ingeniería utilizando herramientas basadas en la Web de las Cosas. Se encuentra ubicado en Beaucheff \#851, Santiago, Chile. El trabajo se realizó de forma presencial en las instalaciones del centro, entre el 3 al 31 de Enero de 2017.}
		

	\subsection{Grupo de trabajo}
	
		\newpar{En el WIC trabajan investigadores de tiempo completo, desarrolladores con experiencia, profesionales del área de la salud debido a que muchos de sus proyectos son en conjunto con la Facultad de Medicina, alumnos de pregrado que pueden ser memorista sobre algún tema de investigación o trabajadores part-time y estudiantes de magister del Departamento de Ingeniería Industrial. En particular en la oficina donde se realizó el trabajo era usada regularmente por 6 personas, de los cuales había un investigador, dos memorista, un ingeniero de proyectos y otro practicante que también trabajó en el \textit{Proyecto AKORI}, que es dirigido por el Ingeniero de Proyecto, Felipe Vera, quien fue el tutor del practicante (pero no se encontraba en la misma oficina).}
\newpage
\subsection{Equipos y Software}	
	
		\subsubsection{Software}
			\newpar{El software utilizado para desarrollar el trabajo fue Python como lenguaje de
			programación (a través del IDLE Pycharm proporcionado por JetBrains \cite{ref2}), el framework para aplicaciones Web Python, Django,  Git para el manejo de control de versiones, el navegador sin interfaz gráfico (conocido como headless browser), PhantomJS, el entorno de pruebas software para aplicaciones web, Selenium, dos bibliotecas de Python, una para edición de imágenes, Python Imaging Library, y otra para la generación de un mapa de colores, Matplotlib, y el sistema operativo Ubuntu. Además cabe descatar del aplicación del \textit{Proyecto AKORI} en la cual se trabajo. A continuación se presenta una breve descripción de los elementos más relevantes para el desarrollo y compresión del trabajo realizado:}
			\begin{enumerate}
				\item \textbf{Python}: Es un lenguaje de programación interpretado cuya filosofía 
				hace hincapié en una sintaxis que favorezca un código legible. Se trata de un 
				lenguaje de programación multiparadigma, ya que soporta orientación a objetos, 
				progrmación imperativa y, en menor medida, programación funcional. En el contexto 
				del trabajo Python sirvió como lenguaje de programación por el lado del servidor 
				para la aplicación web que actualmente funciona como prototipo al \textit{Proyecto AKORI}. 
				\item \textbf{Django}: Es un framework de desarrollo web de código abierto (\textit{open source}), escrito en \textit{Python} que respeta el patrón de diseño conocido como 
				Modelo-Vista-Controlador (MCV), que en pocas palabras separa los componentes más 
				importantes de una aplicación es tres grandes grupos, el modelo donde descansan los
				datos de la aplicación (Base de Datos), la vista que se encarga de todo el aspecto 
				visual de una aplicación, y los controladores que ejecutan toda la lógica de
				funcionamiento. \\ 
				En el caso de \textit{Django} tiene la perculiaridad que las vistas reciben el 
				nombre templates (encargados del \textit{fontend} de la aplicación) y los 
				controladores se llaman vistas (Views en inglés y encargadas del \textit{backend}). 	
				Para el trabajo realizado fue necesario tomar la aplicación desarrollada, anadir
				cambios en los templates y agregar algunas funciones a las vistas para controlar 
				la lógica de la mapa de objetos que en las próximas secciones serán explicadas en 
				detalle.
				\item \textbf{PhantomJS}: Es un navegador sin interfaz gráfica que sirve para 
				realizar pruebas a una aplicación que se encuentra en fase de desarrollo. Fue 
				utilizado para realizar web scrapping, término que describe la recolección de 
				información a través de la Web usando programas automatizados.
				\item \textbf{Selenium}: Es un entorno de pruebas de software para aplicaciones 
				basadas en la web. Se usó para realizar el web scrapping junto a PhantomJs sobre
				los distintas sitios web que fueron testeados.
			\end{enumerate}
	\subsubsection{\textit{Proyecto AKORI}}
	
		\newpar{Para contextualizar de mayor forma el trabajo de práctica realizado a continuación 
		se explicará brevemente que es el \textit{Proyeto AKORI} y cuál es su meta a seguir. }
		
	\subsubsection{Equipo}
	
		\newpar{En cuanto a los equipos computacionales, el centro provee tanto equipos
		estacionarios como portátiles para aquellos que encuentran trabajando e investigando, 
		sin embargo para los prácticantes y memoristas deben llevar sus equipos personales 
		para desarrollar sus labores. En particular la prática realizada solo fue necesario un
		equipo portátil para programar el mapa de objetos.}

	\subsection{Situación Previa}
% NUEVA SECCIÓN
\newpage
\section{Conclusiones}
 	\newpar{Los resultados del trabajo presentado en este informe fueron considerados por el tutor
 	de práctica como satifactorios, siendo puestos en producción al poco tiempo de finalizada la 			práctica, los cuales pueden ser apreaciados en la página oficial del \textit{Proyecto AKORI} 			\cite{ref3}. Además gracias a la programación hecha por el prácticante, cuando se ingresa la url del sitio web a analizar no solo es posible generar un mapa de los objetos más relevantes encerrados por rectangulos, sino que existe un mapa en escala de intensidad de colores de los objetos web más visto a los menos vistos por un usuario (en el sitio web que se está analizando), cambiando la escala en cuatro colores rojo, naranjo, azul y verde.}
\section{Aquí un nuevo tema}
	
	% SUB-SECCIÓN
	\subsection{Haciendo informes como un profesional}
		
		% Se inserta una imagen flotante en la izquierda del documento con \insertimageleft, al igual que las demás funciones, el primer parámetro es opcional, luego viene la ubicación de la imagen, seguido de la escala y por último su leyenda. Para insertar una imagen flotante en la derecha se utiliza \insertimageright usando los mismos parámetros
		\insertimageleft[\label{img:imagen-izquierda}]{ejemplos/test-image-wrap}{0.3}{Apolo flotando a la izquierda.}
		
		\lipsum[1]

		% Párrafos con \newp, lipsum por defecto no añade un párrafo nuevo
		\newp \lipsum[115]
		\newp \lipsum[2]
		
		% Agrega una ecuación con leyendas
		\insertequationcaptioned[\label{eqn:formulasinsentido}]{\int_{a}^{b} f(x) \dd{x} = \fracnpartial{f(x)}{x}{\eta} \cdotp \textstyle \sum_{x=a}^{b} f(x)\cancelto{1+\frac{\epsilon}{k}}{(1+\Delta x)}}{Ecuación sin sentido.}
		
		% Aquí no es necesario usar \newp dado que todas las funciones \insert... añaden un párrafo nuevo por defecto
		\lipsum[115]
		
		% Párrafos con \newp, lipsum por defecto no añade un párrafo nuevo
		\newp \lipsum[4]
		
	% Inserta un subtítulo sin número
	\subsection{Otros párrafos más normales}
	
		% Párrafos con lipsum
		\lipsum[7]
		
		\newp \lipsum[2]
		
		% Se inserta una ecuación larga con el entorno gathered (1 solo número de ecuación)
		\insertgathered[\label{eqn:eqn-larga}]{
			\lpow{\Lambda}{f} = \frac{L\cdot f}{W} \cdot \frac{\pow{\lpow{Q}{e}}{2}}{8 \pow{\pi}{2} \pow{W}{4} g} + \sum_{i=1}^{l} \frac{f \cdot \big( M - d\big)}{l \cdot W} \cdot \frac{\pow{\big(\lpow{Q}{e}- i\cdot Q\big)}{2}}{8 \pow{\pi}{2} \pow{W}{4} g}\\
			Q_e = 2.5Q \cdot \int_{0}^{e} V(x) \dd{x}
		}
	
		% Nuevo párrafo
		\lipsum[4]
		
		% Se inserta un multicols, con esto se pueden escribir en varias columnas
		\begin{multicols}{2}
			
			% Párrafo 1
			\lipsum[4]
			
			% Ecuación encerrada en una caja
			\insertequation[]{ \boxed{f(x) = \fracdpartial{u}{t}} }
			
			% Párrafo 2 del multicols
			\lipsum[1]
			
		\end{multicols}
		
	% SUB-SECCIÓN
	\subsection{Ejemplos de inserción de código fuente}
		
		A continuación se presenta un ejemplo de inserción de código fuente en Python\footnote{El mejor lenguaje del mundo.} (Código \ref{codigo-python}), Java (Código \ref{codigo-java}) y Matlab (Código \ref{codigo-matlab}) utilizando el entorno \texttt{lstlisting}:
		
% Se define el lenguaje del código, cuidado: Los códigos en LaTeX son sensibles a las tabulaciones y espacios en blanco
\begin{lstlisting}[style=Python, caption={Ejemplo en Python.\label{codigo-python}}]
import numpy as np

def incmatrix(genl1, genl2):
	m = len(genl1)
	n = len(genl2)
	M = None # Comentario 1
	VT = np.zeros((n*m, 1), int) # Comentario 2
\end{lstlisting}

\begin{lstlisting}[style=Java, caption={Ejemplo en Java.\label{codigo-java}}]
import java.io.IOException; 
import javax.servlet.*;

// Hola mundo
public class Hola extends GenericServlet {
	public void service(ServletRequest request, ServletResponse response)
	throws ServletException, IOException{
		response.setContentType("text/html");
		PrintWriter pw = response.getWriter();
		pw.println("Hola, mundo!");
		pw.close();
	}
}
\end{lstlisting}

\begin{lstlisting}[style=Matlab, caption={Ejemplo en Matlab.\label{codigo-matlab}}]
% Se crea gráfico
f = figure(1); hold on; movegui(f, 'center');
xlabel('td/Tn'); ylabel('FAD=Umax/Uf0');
title('Espectro de pulso de desplazamiento');

for j = 1:length(BETA)
	fad = ones(1, NDATOS); % Arreglo para el FAD, uno para cada r (o td/Tn)
	
	% Se crea el espectro de respuesta máximo para cada par de beta/r
	for i = 1:NDATOS
		[t, u_t, ~, ~] = main(BETA(j), r(i), M, K, F0, 0);
		fad(i) = max(abs(u_t)) / uf0;
	end
	mx = find(fad == max(fad(:)));
	fprintf('BETA=%.2f, MAX: FAD=%.3f, TD/TN=%.3f\n', BETA(j), fad(mx), tdtn(mx));
	plot(tdtn, fad, 'DisplayName', strcat('\beta=', sprintf('%.2f', BETA(j))));
end
\end{lstlisting}


% NUEVA SECCIÓN
% Inserta una sección sin número
\section{Más ejemplos}
	
	% Inserta un subtítulo sin número
	\subsection{Listas y Enumeraciones}
		
		Hacer listas enumeradas con \LaTeX\ es muy fácil \footnote{También puedes revisar el manual de las enumeraciones en \url{http://www.texnia.com/archive/enumitem.pdf}}, para eso debes usar el comando \texttt{\textbackslash begin\{enumerate\}}, cada elemento empieza por \texttt{\textbackslash item}, resultando:
		
		\begin{enumerate}
			\item Ítem 1
			\item Abracadabra
			\item Manzanas
		\end{enumerate}
		
		También se puede cambiar el tipo de enumeración, se pueden usar letras, números romanos, entre otros. Esto se logra cambiando el \textbf{label} del objeto \texttt{enumerate}. A continuación se muestra un ejemplo usando letras con el estilo \texttt{\textbackslash alph} \footnote{Con \texttt{\textbackslash Alph} las letras aparecen en mayúscula}, números romanos con \texttt{\textbackslash roman} \footnote{Con \texttt{\textbackslash Roman} los números romanos salen en mayúscula} o números griegos con \texttt{\textbackslash greek} \footnote{Una característica propia del template, con \texttt{\textbackslash Greek} las letras griegas están escritas en mayúscula}:
		
		\begin{multicols}{3}
			\begin{enumerate}[label=\alph*) ,font=\bfseries] % Fuente en negrita
				\item Peras
				\item Manzanas
				\item Naranjas
			\end{enumerate}
			
			\begin{enumerate}[label=\greek*) ]
				\item Matemáticas
				\item Lenguaje
				\item Filosofía
			\end{enumerate}
		
			\begin{enumerate}[label=\roman*) ]
				\item Rojo
				\item Café
				\item Morado
			\end{enumerate}
		\end{multicols}
		
		Para hacer listas sin numerar con \LaTeX\ hay que usar el comando \texttt{\textbackslash begin\{itemize\}}, cada elemento empieza por \texttt{\textbackslash item}, resultando:
		
		\begin{multicols}{3}
			\begin{itemize}[label={--}]
				\item Peras
				\item Manzanas
				\item Naranjas
			\end{itemize}
			
			\begin{enumerate}[label={*}]
				\item Rojo
				\item Café
				\item Morado
			\end{enumerate}
			
			\begin{itemize}
				\item Árboles
				\item Pasto
				\item Flores
			\end{itemize}
		\end{multicols}
		
	% Inserta un subtítulo sin número
	\subsection{Otros}
		
		Recuerda revisar el manual de todas las funciones de este template visitando el siguiente link: \url{http://ppizarror.com/Template-Informe/}. Además si necesitas una ayuda muy específica sobre el template me puedes enviar un correo a \insertemail{pablo.pizarro@ing.uchile.cl}.

% ANEXO
\newpage
\begin{anexo}
	\section{Cálculos realizados}
	
		\lipsum[69]
		
		% Imagen, se numerará automáticamente con la letra del anexo
		\insertimage[\label{img:anexo-2}]{ejemplos/test-image.png}{scale=0.15}{Imagen en anexo.}
		
		\lipsum[10]
		
		% Tablas
		\begin{table}[htbp]
			\centering
			\caption{Tabla de cálculo.}
			\begin{tabular}{ccc}
				\hline
				\textbf{Elemento} & $\epsilon_i$ & \boldmath{}\textbf{Valor}\unboldmath{} \bigstrut\\
				\hline
				A     & 10    & 3,14$\pi$ \bigstrut[t]\\
				B     & 20    & 6 \\
				C     & 30    & 7 \\
				\end{tabular}
			\label{tab:anexo-1}
		\end{table}
	
	\newpage
	\section{Más cálculos}
	
		% Párrafo
		\lipsum[1]\newp\lipsum[4]
		
		% Tabla de encuestas
		\begin{table}[htbp]
			\centering
			\caption{Resultados encuesta.}
			\begin{tabular}{ccc}
				\hline
				\textbf{Herramienta} & \textbf{Nota} & \textbf{Recomendado} \bigstrut\\
				\hline
				Word  & 0\%   & No $\frownie$\\
				\LaTeX & 100\% & Si $\checkmark$ \\
			\end{tabular}
			\label{tab:anexo-2}
		\end{table}
		
\end{anexo}

% REFERENCIAS (ESTILO BIBTEX)
\newpage % Salto de página
\begin{references}
	\bibitem{ref1}
	\textit{Sitio Web del WIC}
	\url{http://www.wic.uchile.cl}
	
	\bibitem{ref2}
	\textit{PyCharm IDLE}
	\url{https://www.jetbrains.com/pycharm/}
	
	\bibitem{ref3}
	\textit{Página Oficial del Proyecto AKORI} 
	\url{https://www.akori-project.cl}
\end{references}
