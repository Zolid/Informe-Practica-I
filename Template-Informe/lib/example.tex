% Template:     Informe/Reporte LaTeX
% Documento:    Archivo de ejemplo
% Versión:      4.3.6 (30/07/2017)
% Codificación: UTF-8
%
% Autor: Pablo Pizarro R.
%        Facultad de Ciencias Físicas y Matemáticas
%        Universidad de Chile
%        pablo.pizarro@ing.uchile.cl, ppizarror.com
%
% Manual template: [http://latex.ppizarror.com/Template-Informe/]
% Licencia MIT:    [https://opensource.org/licenses/MIT/]

% NUEVA SECCIÓN
% Las secciones se inician con \section, si se quiere una sección sin "número" se pueden usar las funciones \sectionanum (sección sin número) o la función \sectionanumnoi para crear el mismo título sin numerar y sin aparecer en el índice
\section{Introducción}
	
	% SUB-SECCIÓN
	% Las sub-secciones se inician con \subsection, si se quiere una sub-sección sin "número" se pueden usar las funciones \subsectionanum (nuevo subtítulo sin numeración) o la función \subsectionanumnoi para crear el mismo subtítulo sin numerar y sin aparecer en el índice
	\subsection{Lugar de Trabajo}
		
		\newpar{La prática profesional se realizó en el Web Intelligence Centre \cite{ref1}, en adelante WIC, es un centro de investigación depediente de la Facultad de Ciencias Físicas y Matemáticas de la Universidad de Chile, cuya misión es desarrollar investigación en el campo de Tecnologías de Información creando soluciones para abordar problemas complejos de ingeniería utilizando herramientas basadas en la Web de las Cosas. Se encuentra ubicado en Beaucheff \#851, Santiago, Chile. El trabajo se realizó de forma presencial en las instalaciones del centro, entre el 3 al 31 de Enero de 2017.}
		

	\subsection{Grupo de trabajo}
	
		\newpar{En el WIC trabajan investigadores de tiempo completo, desarrolladores con experiencia, profesionales del área de la salud debido a que muchos de sus proyectos son en conjunto con la Facultad de Medicina, alumnos de pregrado que pueden ser memorista sobre algún tema de investigación o trabajadores part-time y estudiantes de magister del Departamento de Ingeniería Industrial. En particular en la oficina donde se realizó el trabajo era usada regularmente por 6 personas, de los cuales había un investigador, dos memorista, un ingeniero de proyectos y otro practicante que también trabajó en el \textit{Proyecto AKORI}, que es dirigido por el Ingeniero de Proyecto, Felipe Vera, quien fue el tutor del practicante (pero no se encontraba en la misma oficina).}
\newpage
\subsection{Equipos y Software}	
	
		\subsubsection{Software}
			\newpar{El software utilizado para desarrollar el trabajo fue \textit{Python} como lenguaje de programación (a través del IDLE Pycharm proporcionado por JetBrains \cite{ref2}), el framework para aplicaciones Web Python, \textit{Django}, \textit{Javascript} y \textit{AJAX} para la creación de páginas web dinámicas, \textit{Git} para el manejo de control de versiones, el navegador sin interfaz gráfico (conocido como headless browser), PhantomJS, el entorno de pruebas software para aplicaciones web, Selenium, dos bibliotecas de Python, una para edición de imágenes, Python Imaging Library, y otra para la generación de un mapa de colores, Matplotlib, y el sistema operativo Ubuntu. Además cabe descatar del aplicación del \textit{Proyecto AKORI} en la cual se trabajo. A continuación se presenta una breve descripción de los elementos más relevantes para el desarrollo y compresión del trabajo realizado:}
			\begin{enumerate}
				\item \textbf{Python}: Es un lenguaje de programación interpretado cuya filosofía 
				hace hincapié en una sintaxis que favorezca un código legible. Se trata de un 
				lenguaje de programación multiparadigma, ya que soporta orientación a objetos, 
				progrmación imperativa y, en menor medida, programación funcional. En el contexto 
				del trabajo Python sirvió como lenguaje de programación por el lado del servidor 
				para la aplicación web que actualmente funciona como prototipo al \textit{Proyecto AKORI}. 
				\item \textbf{Django}: Es un framework de desarrollo web de código abierto (\textit{open source}), escrito en \textit{Python} que respeta el patrón de diseño conocido como 
				Modelo-Vista-Controlador (MCV), que en pocas palabras separa los componentes más 
				importantes de una aplicación es tres grandes grupos, el modelo donde descansan los
				datos de la aplicación (Base de Datos), la vista que se encarga de todo el aspecto 
				visual de una aplicación, y los controladores que ejecutan toda la lógica de
				funcionamiento. \\ 
				En el caso de \textit{Django} tiene la perculiaridad que las vistas reciben el 
				nombre templates (encargados del \textit{fontend} de la aplicación) y los 
				controladores se llaman vistas (Views en inglés y encargadas del \textit{backend}). 	
				Para el trabajo realizado fue necesario tomar la aplicación desarrollada, anadir
				cambios en los templates y agregar algunas funciones a las vistas para controlar 
				la lógica de la mapa de objetos que en las próximas secciones serán explicadas en 
				detalle.
				\item \textbf{Javascript} y \textbf{AJAX}: Es un lenguaje de programación que 
				es utilizado para la creación de páginas web de dinámicas por el lado del cliente
				en una aplicación web. \textit{AJAX} es una extensión a la funcionalidad de 
				\textit{Javascript} que hace llamadas asincronas al servidor web obteniendo 
				información sin recargar las página por completo. Dentro del trabajo de prática
				fueron utilizados para obtener el input de datos para el despliegue del mapa de 
				objetos sin recargar toda la aplicación en la consulta. 
				\item \textbf{PhantomJS}: Es un navegador sin interfaz gráfica que sirve para 
				realizar pruebas a una aplicación que se encuentra en fase de desarrollo. Fue 
				utilizado para realizar web scrapping, término que describe la recolección de 
				información a través de la Web usando programas automatizados.
				\item \textbf{Selenium}: Es un entorno de pruebas de software para aplicaciones 
				basadas en la web. Se usó para realizar el web scrapping junto a PhantomJs sobre
				los distintas sitios web que fueron testeados.
			\end{enumerate}
	\subsubsection{\textit{Proyecto AKORI}}
	
		\newpar{Para contextualizar de mayor forma el trabajo de práctica realizado a continuación 
		se explicará brevemente que es el \textit{Proyeto AKORI} y cuál es su meta a seguir. \\ 
		Como meta general es desarrollar una plataforma virtual que permita extrapolar el 
		comportamiento de navegación cuandos los usuarios ingresan a sitio web a partir de la 
		extracción de patrones de datos analizados con técnicas de minería de datos, eye tracking, 
		y encefalografía. Para lograr esto, el proyecto se divide en 5 grandes etapas que son las 
		siguientes:
			\begin{enumerate}
				\item Adaptar y refinar un repositorio de datos elaborado con estímulos visuales obtenidos desde sitios web, incorporando los datos de un mayor número de personas y diferenciándolos según variables sociodemográficas.
				\item Adaptar e integrar algoritmos de minería de datos masivos (big data) para determinar patrones que permitan caracterizar y extrapolar la navegación de un usuario en un sitio web mediante análisis de exploración visual utilizando datos provenientes desde eye-tracking, dilatación pupilar y electroencefalografía.
				\item Diseñar, construir y evaluar una plataforma prototipo que integre algoritmos de web Intelligence, un repositorio de datos en base a estímulos visuales y electroencefalografía con patrones de comportamiento web para extrapolar las preferencias y el comportamiento de potenciales usuarios web.
				\item Testear y evaluar funcionalidades del sitio, capacidad del sistema para lograr lo que se desea y minimizar errores de uso; además de medir el nivel de usabilidad (facilidad de aprender y recordar, eficiencia, minimización de errores de uso y satisfacción del usuario).
				\item Prospectar y valorizar el mercado, como también la propiedad intelectual. Definir una estrategia para el empaquetamiento y transferencia de la tecnología.
			\end{enumerate}
			A pesar de que las investigaciones continúan, existe un propotipo de la aplicación web
			con algunos módulos (mapas de estadísticos) en desarrollo y otros ya implementados. A continuación se escriben los modulos ya implementados, incluido el que desarrollo el practicante.
				\begin{enumerate}
					\item \textbf{Mapa de Fijación Ocular}: Contruye un mapa de calor con las zonas
					más probables de fijación ocular en un sitio web a partir de una escala de 
					colores los clientes pueden visualizar las zonas mas vistas en rojo, las menos 
					vistas en azul y las que tiene ninguna probabilidad de ser vistas adquieren un 
					color en escala de grises.
					\item \textbf{Mapa de Dilatación Pupilar}: Construye un mapa calor siguiendo una
					escala similar al mapa de fijación ocular, con la diferencia que las zonas que 
					adquieren colores más intensos muestra donde es más probable que un usuario haga
					un click dentro de la página analizada.
					\item \textbf{Mapa de Objetos de Claves}: Se construye un mapa que muestra los
					objetos web más relevantes que un usuario visualiza al ingresar a un sitio web, 
					los objetos más importantes son coloreados con más oscuros que los menos vistos. 
				\end{enumerate}
			Además de los módulos explicados anteriormente, se está trabajando en tres mapas más 
			que aún no han sido llevados a producción estos son, el índice de claridad, el mapa de 
			probabilidad de ticks, y el mapa de percepción; donde los primeros dos son complementos
			a los ya implementados, y el último mostrará a donde es más probable que el usuario 
			dirigirá la mirada los primeros tres segundos de ingresar al sitio web.				 
			}
		
	\subsubsection{Equipo}
	
		\newpar{En cuanto a los equipos computacionales, el centro provee tanto equipos
		estacionarios como portátiles para aquellos que encuentran trabajando e investigando, 
		sin embargo para los prácticantes y memoristas deben llevar sus equipos personales 
		para desarrollar sus labores. En particular la prática realizada solo fue necesario un
		equipo portátil para programar el mapa de objetos.}

	\subsection{Situación Previa}
% NUEVA SECCIÓN
\newpage
\section{Conclusiones}
 	\newpar{Los resultados del trabajo presentado en este informe fueron considerados por el tutor
 	de práctica como satifactorios, siendo estos puestos en producción al poco tiempo de finalizada la 			práctica, los cuales pueden ser apreciados en la página oficial del \textit{Proyecto AKORI} 			\cite{ref3}. Además gracias a la programación hecha por el prácticante cuando se ingresa la url del sitio web a analizar no solo es posible generar un mapa de los objetos más relevantes encerrados por rectangulos, sino que existe un mapa en escala de intensidad de colores de los objetos web más a los menos vistos por un usuario (en el sitio web que se está analizando), cambiando la escala en cuatro colores rojo, naranjo, azul y verde. \\ \\ Respecto al futuro de la aplicación
 	hay que destacar que aún se encuentra trabajando en ella realizando actulizaciones para mejorar su llegada al público o a los posible clientes (si llega a comercializarse) en términos de 
 	visualizaciones que le provean más y mejor información al usuario sobre el sitio web a analizar, 
 	en particular se seguirá trabajando en las secciones modificadas y agregadas por el practicante,
 	en miras de mejorar la performance de las visualizaciones de los elementos más vistos dentro 
 	del DOM en el mapa de objetos mediante optimizaciones en la extracción de datos y algoritmos
 	de búsqueda de objetos web más eficientes. \\ \\ Dentro del aprendizaje obtenido en el proceso
 	de práctica se destacan a nivel técnico el aprendizaje del framework \textit{Django} para el desarrollo web y sus elementos específicos como vistas, modelos, templates. También el aprendizaje 
 	de \textit{Javascritp} es especialmente el uso de \textit{AJAX} para el despligue del mapa de
 	objetos sin actualizar la página completa y \textit{Git} para el manejo de control de versiones.
 	Otro punto no menos importante dentro del aprendizajes de todas las tecnologías fue lo 
 	primordial que es leer la documentación antes de utilizar cualquier tecnología ya que 
 	simplifica mucho el tiempo de aprendizaje. Dentro de otros aspectos se aprendió sobre el
 	trabajo en equipo, la importancia de mantener constantemente comunicación para realizar un 
 	buen trabajo en particular cuando se trabaja en un equipo multidisciplinario. El practicante
 	estuvo en constante contacto con algunos investigadores y el con encargado para obtener un 
 	mapa de objeto acorde a los requerimientos y entendible para cualquier usuario use la 
 	plataforma del \textit{Proyecto AKORI}. Finalmente queda destacar el aprendizaje de documentar
 	bien el código hecho para que sea fácil de entender y extender a medida que el proyecto crece, 
 	y además de contar con buen diseño en el código muchas veces esto puede resultar fácil, pero
 	en la práctica puede ser determinante la contuinidad de una aplicación robusta que pueda seguir
 	creciendo y no empezar un sistema que prácticamente hará lo mismo desde 0.}

\newpage
\section{Aquí un nuevo tema}
	
	% SUB-SECCIÓN
	\subsection{Haciendo informes como un profesional}
		
		% Se inserta una imagen flotante en la izquierda del documento con \insertimageleft, al igual que las demás funciones, el primer parámetro es opcional, luego viene la ubicación de la imagen, seguido de la escala y por último su leyenda. Para insertar una imagen flotante en la derecha se utiliza \insertimageright usando los mismos parámetros
		\insertimageleft[\label{img:imagen-izquierda}]{ejemplos/test-image-wrap}{0.3}{Apolo flotando a la izquierda.}
		
		\lipsum[1]

		% Párrafos con \newp, lipsum por defecto no añade un párrafo nuevo
		\newp \lipsum[115]
		\newp \lipsum[2]
		
		% Agrega una ecuación con leyendas
		\insertequationcaptioned[\label{eqn:formulasinsentido}]{\int_{a}^{b} f(x) \dd{x} = \fracnpartial{f(x)}{x}{\eta} \cdotp \textstyle \sum_{x=a}^{b} f(x)\cancelto{1+\frac{\epsilon}{k}}{(1+\Delta x)}}{Ecuación sin sentido.}
		
		% Aquí no es necesario usar \newp dado que todas las funciones \insert... añaden un párrafo nuevo por defecto
		\lipsum[115]
		
		% Párrafos con \newp, lipsum por defecto no añade un párrafo nuevo
		\newp \lipsum[4]
		
	% Inserta un subtítulo sin número
	\subsection{Otros párrafos más normales}
	
		% Párrafos con lipsum
		\lipsum[7]
		
		\newp \lipsum[2]
		
		% Se inserta una ecuación larga con el entorno gathered (1 solo número de ecuación)
		\insertgathered[\label{eqn:eqn-larga}]{
			\lpow{\Lambda}{f} = \frac{L\cdot f}{W} \cdot \frac{\pow{\lpow{Q}{e}}{2}}{8 \pow{\pi}{2} \pow{W}{4} g} + \sum_{i=1}^{l} \frac{f \cdot \big( M - d\big)}{l \cdot W} \cdot \frac{\pow{\big(\lpow{Q}{e}- i\cdot Q\big)}{2}}{8 \pow{\pi}{2} \pow{W}{4} g}\\
			Q_e = 2.5Q \cdot \int_{0}^{e} V(x) \dd{x}
		}
	
		% Nuevo párrafo
		\lipsum[4]
		
		% Se inserta un multicols, con esto se pueden escribir en varias columnas
		\begin{multicols}{2}
			
			% Párrafo 1
			\lipsum[4]
			
			% Ecuación encerrada en una caja
			\insertequation[]{ \boxed{f(x) = \fracdpartial{u}{t}} }
			
			% Párrafo 2 del multicols
			\lipsum[1]
			
		\end{multicols}
		
	% SUB-SECCIÓN
	\subsection{Ejemplos de inserción de código fuente}
		
		A continuación se presenta un ejemplo de inserción de código fuente en Python\footnote{El mejor lenguaje del mundo.} (Código \ref{codigo-python}), Java (Código \ref{codigo-java}) y Matlab (Código \ref{codigo-matlab}) utilizando el entorno \texttt{lstlisting}:
		
% Se define el lenguaje del código, cuidado: Los códigos en LaTeX son sensibles a las tabulaciones y espacios en blanco
\begin{lstlisting}[style=Python, caption={Ejemplo en Python.\label{codigo-python}}]
import numpy as np

def incmatrix(genl1, genl2):
	m = len(genl1)
	n = len(genl2)
	M = None # Comentario 1
	VT = np.zeros((n*m, 1), int) # Comentario 2
\end{lstlisting}

\begin{lstlisting}[style=Java, caption={Ejemplo en Java.\label{codigo-java}}]
import java.io.IOException; 
import javax.servlet.*;

// Hola mundo
public class Hola extends GenericServlet {
	public void service(ServletRequest request, ServletResponse response)
	throws ServletException, IOException{
		response.setContentType("text/html");
		PrintWriter pw = response.getWriter();
		pw.println("Hola, mundo!");
		pw.close();
	}
}
\end{lstlisting}

\begin{lstlisting}[style=Matlab, caption={Ejemplo en Matlab.\label{codigo-matlab}}]
% Se crea gráfico
f = figure(1); hold on; movegui(f, 'center');
xlabel('td/Tn'); ylabel('FAD=Umax/Uf0');
title('Espectro de pulso de desplazamiento');

for j = 1:length(BETA)
	fad = ones(1, NDATOS); % Arreglo para el FAD, uno para cada r (o td/Tn)
	
	% Se crea el espectro de respuesta máximo para cada par de beta/r
	for i = 1:NDATOS
		[t, u_t, ~, ~] = main(BETA(j), r(i), M, K, F0, 0);
		fad(i) = max(abs(u_t)) / uf0;
	end
	mx = find(fad == max(fad(:)));
	fprintf('BETA=%.2f, MAX: FAD=%.3f, TD/TN=%.3f\n', BETA(j), fad(mx), tdtn(mx));
	plot(tdtn, fad, 'DisplayName', strcat('\beta=', sprintf('%.2f', BETA(j))));
end
\end{lstlisting}


% NUEVA SECCIÓN
% Inserta una sección sin número
\section{Más ejemplos}
	
	% Inserta un subtítulo sin número
	\subsection{Listas y Enumeraciones}
		
		Hacer listas enumeradas con \LaTeX\ es muy fácil \footnote{También puedes revisar el manual de las enumeraciones en \url{http://www.texnia.com/archive/enumitem.pdf}}, para eso debes usar el comando \texttt{\textbackslash begin\{enumerate\}}, cada elemento empieza por \texttt{\textbackslash item}, resultando:
		
		\begin{enumerate}
			\item Ítem 1
			\item Abracadabra
			\item Manzanas
		\end{enumerate}
		
		También se puede cambiar el tipo de enumeración, se pueden usar letras, números romanos, entre otros. Esto se logra cambiando el \textbf{label} del objeto \texttt{enumerate}. A continuación se muestra un ejemplo usando letras con el estilo \texttt{\textbackslash alph} \footnote{Con \texttt{\textbackslash Alph} las letras aparecen en mayúscula}, números romanos con \texttt{\textbackslash roman} \footnote{Con \texttt{\textbackslash Roman} los números romanos salen en mayúscula} o números griegos con \texttt{\textbackslash greek} \footnote{Una característica propia del template, con \texttt{\textbackslash Greek} las letras griegas están escritas en mayúscula}:
		
		\begin{multicols}{3}
			\begin{enumerate}[label=\alph*) ,font=\bfseries] % Fuente en negrita
				\item Peras
				\item Manzanas
				\item Naranjas
			\end{enumerate}
			
			\begin{enumerate}[label=\greek*) ]
				\item Matemáticas
				\item Lenguaje
				\item Filosofía
			\end{enumerate}
		
			\begin{enumerate}[label=\roman*) ]
				\item Rojo
				\item Café
				\item Morado
			\end{enumerate}
		\end{multicols}
		
		Para hacer listas sin numerar con \LaTeX\ hay que usar el comando \texttt{\textbackslash begin\{itemize\}}, cada elemento empieza por \texttt{\textbackslash item}, resultando:
		
		\begin{multicols}{3}
			\begin{itemize}[label={--}]
				\item Peras
				\item Manzanas
				\item Naranjas
			\end{itemize}
			
			\begin{enumerate}[label={*}]
				\item Rojo
				\item Café
				\item Morado
			\end{enumerate}
			
			\begin{itemize}
				\item Árboles
				\item Pasto
				\item Flores
			\end{itemize}
		\end{multicols}
		
	% Inserta un subtítulo sin número
	\subsection{Otros}
		
		Recuerda revisar el manual de todas las funciones de este template visitando el siguiente link: \url{http://ppizarror.com/Template-Informe/}. Además si necesitas una ayuda muy específica sobre el template me puedes enviar un correo a \insertemail{pablo.pizarro@ing.uchile.cl}.

% ANEXO
\newpage
\begin{anexo}
	\section{Cálculos realizados}
	
		\lipsum[69]
		
		% Imagen, se numerará automáticamente con la letra del anexo
		\insertimage[\label{img:anexo-2}]{ejemplos/test-image.png}{scale=0.15}{Imagen en anexo.}
		
		\lipsum[10]
		
		% Tablas
		\begin{table}[htbp]
			\centering
			\caption{Tabla de cálculo.}
			\begin{tabular}{ccc}
				\hline
				\textbf{Elemento} & $\epsilon_i$ & \boldmath{}\textbf{Valor}\unboldmath{} \bigstrut\\
				\hline
				A     & 10    & 3,14$\pi$ \bigstrut[t]\\
				B     & 20    & 6 \\
				C     & 30    & 7 \\
				\end{tabular}
			\label{tab:anexo-1}
		\end{table}
	
	\newpage
	\section{Más cálculos}
	
		% Párrafo
		\lipsum[1]\newp\lipsum[4]
		
		% Tabla de encuestas
		\begin{table}[htbp]
			\centering
			\caption{Resultados encuesta.}
			\begin{tabular}{ccc}
				\hline
				\textbf{Herramienta} & \textbf{Nota} & \textbf{Recomendado} \bigstrut\\
				\hline
				Word  & 0\%   & No $\frownie$\\
				\LaTeX & 100\% & Si $\checkmark$ \\
			\end{tabular}
			\label{tab:anexo-2}
		\end{table}
		
\end{anexo}

% REFERENCIAS (ESTILO BIBTEX)
\newpage % Salto de página
\begin{references}
	\bibitem{ref1}
	\textit{Sitio Web del WIC}
	\url{http://www.wic.uchile.cl}
	
	\bibitem{ref2}
	\textit{PyCharm IDLE}
	\url{https://www.jetbrains.com/pycharm/}
	
	\bibitem{ref3}
	\textit{Página Oficial del Proyecto AKORI} 
	\url{https://www.akori-project.cl}
\end{references}
